\usepackage{hyperref}
\usepackage{cleveref}
\usepackage{amsthm}
\usepackage{multirow}
\usepackage{array}
\usepackage{rotating}

\def\TVaR{\mathsf{TVaR}}
\def\cov{\mathsf{cov}}
\newcommand{\E}{\mathsf{E}}  % New command for \E
\renewcommand{\Pr}{\mathsf{P}}  % Redefine \Pr
\newcommand{\esssup}{\operatorname{ess\,sup}}

\newtheorem{definition}{Definition}


%==============================
% tikz and related
%==============================
\usepackage{amsmath}
\usepackage{amssymb}         % If needed

\usepackage{multirow}
\usepackage{url}
\usepackage{tikz}
\usepackage{color}

\usetikzlibrary{arrows,calc,positioning,shadows.blur,decorations.pathreplacing}
\usetikzlibrary{automata}
\usetikzlibrary{fit}
\usetikzlibrary{snakes}
\usetikzlibrary{intersections}
\usetikzlibrary{decorations.markings,decorations.text, decorations.pathmorphing,decorations.shapes}
\usetikzlibrary{decorations.fractals,decorations.footprints}
\usetikzlibrary{graphs}
\usetikzlibrary{matrix}
\usetikzlibrary{shapes.geometric}
\usetikzlibrary{mindmap, shadows}
\usetikzlibrary{backgrounds}
\usetikzlibrary{cd}

\newcommand{\grtspacer}{\vphantom{lp}}
\newcommand{\I}{\vphantom{lp}}


%==============================
% Float placement customization
%==============================
\usepackage{float} % Required for \floatplacement and [H] (force here)

% Default float placement preferences:
% These set global default positions for all floats of the given type.

\floatplacement{table}{h}   % Try to place tables "here" by default
\floatplacement{figure}{t}  % Try to place figures at the top of the page

% Notes on float specifiers:
%   h  = here (if LaTeX thinks it fits)
%   t  = top of page
%   b  = bottom of page
%   p  = on a separate float-only page
%   H  = exactly here (requires \usepackage{float})

% You can still override per float:
% \begin{table}[htbp]   % Try here, then top, bottom, float page
% \begin{figure}[H]     % Force placement exactly here


%==============================
% Listing formats
%==============================
% Suppress Quarto's default codelisting environment definition
% This prevents conflicts with listings package's captioning

% \makeatletter
% \@ifundefined{newfloat}{\@namedef{newfloat}{#1}}{\@namedef{newfloat}#1}
% \makeatother

% \makeatletter
% \def\newfloat#1[#2]#3#4{%
%   \newenvironment{#1}{\relax}{\relax}%
% }
% \makeatother
% \usepackage{codelisting}

\usepackage{caption}

% Choose one of the following font packages:
% \usepackage{beramono} % Example: Bera Mono
% \usepackage{inconsolata} % Example: Inconsolata
% \usepackage{FiraMono}    % Example: Fira Mono (requires specific font setup)
\usepackage{sourcecodepro} % Example: Source Code Pro

\usepackage{xcolor} % Often needed for more robust color definitions, ensure it's loaded
\usepackage{listings}

\definecolor{lightgray}{rgb}{0.96,0.96,0.96} % Define the color

\definecolor{backgroundlightgray}{rgb}{0.96,0.96,0.96} % Very light background
\definecolor{keywordcolor}{rgb}{0.0, 0.0, 0.5}       % Dark Blue for keywords
\definecolor{stringcolor}{rgb}{0.6, 0.0, 0.0}        % Dark Red/Maroon for strings
\definecolor{commentcolor}{rgb}{0.35, 0.35, 0.35}    % Medium-Dark Grey for comments

% size: \tiny, \scriptsize, \footnotesize, \small, \normalsize
\lstset{
    basicstyle=\footnotesize\ttfamily,
    % basicstyle=\scriptsize\ttfamily,
    keywordstyle=\color{keywordcolor},
    stringstyle=\color{stringcolor},
    commentstyle=\color{commentcolor},
    breaklines=true,
    showstringspaces=false,
    frame=single,
    framerule=0pt,
    framesep=5pt,
    backgroundcolor=\color{backgroundlightgray},
}


% \lstset{
%     % Uses footnotesize and typewriter font
%     basicstyle=\scriptsize\ttfamily,
%     % Add other listings options here if needed, e.g.,
%     % numbers=left,
%     % numberstyle=\tiny,
%     keywordstyle=\color{blue},
%     stringstyle=\color{red},
%     commentstyle=\color{green},
%     breaklines=true,
%     showstringspaces=false,
%     frame=none,
%     backgroundcolor=\color{lightgray}, % Apply it here
%     % captionpos=b, % bottom captions
%     frame=single,      % Create a frame (even if invisible)
%     framerule=0pt,     % Make the frame invisible
%     framesep=5pt,      % Padding around the content inside the frame
%     % Or more granular control:
%     % xleftmargin=5pt,
%     % xrightmargin=5pt,
%     % xtopmargin=5pt,
%     % xbottommargin=5pt,
% }



%% VERSION 2 solution from Gemini
% macros.tex
% \usepackage{minted} % Good practice if you use \setminted
% \setminted{
%     fontsize=\footnotesize, % This controls the font size of the echoed code
%     frame=none,             % This removes the frame/box
%     % Other minted options you might want:
%     % linenumbers,
%     % breaklines=true,
%     % bgcolor=white, % If you want a background color
% }
